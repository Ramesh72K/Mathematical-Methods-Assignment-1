\documentclass[]{book}

%These tell TeX which packages to use.
\usepackage{array,epsfig}
\usepackage{amsmath}
\usepackage{amsfonts}
\usepackage{amssymb}
\usepackage{amsxtra}
\usepackage{amsthm}
\usepackage{mathrsfs}
\usepackage{color}
\usepackage{graphicx}
\usepackage{xcolor}
\raggedbottom


%Here I define some theorem styles and shortcut commands for symbols I use often
\theoremstyle{definition}
\newtheorem{defn}{Definition}
\newtheorem{thm}{Theorem}
\newtheorem{cor}{Corollary}
\newtheorem*{rmk}{Remark}
\newtheorem{lem}{Lemma}
\newtheorem*{joke}{Joke}
\newtheorem{ex}{Example}
\newtheorem*{soln}{Solution}
\newtheorem{prop}{Proposition}

\newcommand{\lra}{\longrightarrow}
\newcommand{\ra}{\rightarrow}
\newcommand{\surj}{\twoheadrightarrow}
\newcommand{\graph}{\mathrm{graph}}
\newcommand{\bb}[1]{\mathbb{#1}}
\newcommand{\Z}{\bb{Z}}
\newcommand{\Q}{\bb{Q}}
\newcommand{\R}{\bb{R}}
\newcommand{\C}{\bb{C}}
\newcommand{\N}{\bb{N}}
\newcommand{\M}{\mathbf{M}}
\newcommand{\m}{\mathbf{m}}
\newcommand{\MM}{\mathscr{M}}
\newcommand{\HH}{\mathscr{H}}
\newcommand{\Om}{\Omega}
\newcommand{\Ho}{\in\HH(\Om)}
\newcommand{\bd}{\partial}
\newcommand{\del}{\partial}
\newcommand{\bardel}{\overline\partial}
\newcommand{\textdf}[1]{\textbf{\textsf{#1}}\index{#1}}
\newcommand{\img}{\mathrm{img}}
\newcommand{\ip}[2]{\left\langle{#1},{#2}\right\rangle}
\newcommand{\inter}[1]{\mathrm{int}{#1}}
\newcommand{\exter}[1]{\mathrm{ext}{#1}}
\newcommand{\cl}[1]{\mathrm{cl}{#1}}
\newcommand{\ds}{\displaystyle}
\newcommand{\vol}{\mathrm{vol}}
\newcommand{\cnt}{\mathrm{ct}}
\newcommand{\osc}{\mathrm{osc}}
\newcommand{\LL}{\mathbf{L}}
\newcommand{\UU}{\mathbf{U}}
\newcommand{\support}{\mathrm{support}}
\newcommand{\AND}{\;\wedge\;}
\newcommand{\OR}{\;\vee\;}
\newcommand{\Oset}{\varnothing}
\newcommand{\st}{\ni}
\newcommand{\wh}{\widehat}

%Pagination stuff.
\setlength{\topmargin}{-.3 in}
\setlength{\oddsidemargin}{0in}
\setlength{\evensidemargin}{0in}
\setlength{\textheight}{9.in}
\setlength{\textwidth}{6.5in}
\pagestyle{empty}



\begin{document}


\begin{center}
{\textbf{\Large AE601 Mathematical Methods in Aerospace Engineering}}\\

\normalsize{Submitted by Ramesh M (SC23M061)}\\
\end{center}

\vspace{0.2 cm}

\begin{enumerate}
%1st Problem start
\item
Find the  general solution of the given second-order differential equation
\[
12y'' - 5y' - 2y = 0
\]
\emph{\textbf{Solution:}}
The characteristic equation is

\[12\lambda^2 - 5\lambda - 2 = 0\]

Solving the quadratic equation, we get \(\lambda_1 = 0.6667\) and \(\lambda_2 = -0.25\).

Since the roots are real and distinct, the general solution is

\[{y}(x) = Ae^{0.6667x} + Be^{-0.25x}\]
%1st Problem ends

%2nd Problem starts
\item 
Solve the initial value problem
\[y'' + y' - 6y = 0,\]
\[
y(0) = 10,\;y'(0) = 0
\]
and plot the solution.\\
\\
\emph{\textbf{Solution:}}

The characteristic equation is
\[
\lambda^2 + \lambda - 6 = 0
\]

Solving the quadratic equation, we get \(\lambda_1 = 2\) and \(\lambda_2 = -3\).
Since the roots are real and distinct, the general solution is
\[
y = Ae^{2x} + Be^{-3x}
\]
To find the constants A and B, we substitute the initial conditions
\[
A = 6\:and\:B = 4
\]
The general solution is
\[{y}(x) = 6e^{2x} + 4e^{-3x}\]
\begin{figure}[htbp]
    \centering
    \includegraphics[width=0.8\textwidth]{Images/Problem 2.jpg}
    \caption{Plot of $y(x) = 6e^{2x} + 4e^{-3x}$}
\end{figure}
\pagebreak
%2nd Problem ends

%3rd Problem starts
\item 
Solve the initial value problem
\[x^{2}{y}'' + x{y}' - 4y = 0\]
\[
y(1) = 2, \; {y}'(1) = 0
\]
and plot the solution.\\
\\
\emph{\textbf{Solution:}}

Comparing our equation with the Euler-Cauchy equation
\[
x^2{y}'' + x{y}' - 4y = 0 \;and\; x^2{y}'' + ax{y}' - by = 0
\]
\[
a = 1,\;b = -4
\]
The auxiliary equation is
\[
m^2 + (a-1)m + b = 0
\]
\[
m^2 + (0)m - 4 = 0
\]
solving the quadratic equation, we get 
\[m_1 = 2\;and\;m_2 = -2\]
The general solution is
\[
y = Ax^{2} + Bx^{-2}
\]
substituting the initial conditions, 
\[
A = 1\:and\:B = 1
\]
Substituting A and B, we get the general solution
\[
y(x) = x^{2} + x^{-2}
\]
\begin{figure}[htbp]
    \centering
    \includegraphics[width=0.8\textwidth]{Images/Problem 3.jpg}
    \caption{Plot of $y(x) = x^{2} + x^{-2}$}
\end{figure}
\pagebreak
%End of 3rd problem

%4th Problem start
\item
{Two roots of a cubic auxiliary equation with real coefficients are $\lambda_1 = -0.5$ and $\lambda_2 = 3 + i$. What is the corresponding homogeneous linear differential equation? Is your answer unique?}\\

\emph{\textbf{Solution:}}
We know that the complex roots come in conjugate pairs, so $\lambda_3 = 3 - i$.\\
To get the auxiliary equation,
\[
(\lambda + 0.5)(\lambda - (3 + i))(\lambda + (3 - i)) = 0
\]
Multiplying the factors to get the cubic auxiliary equation,
\[
\lambda^3 - 5.5\lambda^2 + 7\lambda + 5 = 0
\]
The corresponding homogeneous linear differential equation is,
\[
{y}''' - 5.5{y}'' + 7{y}' + 5{y} = 0
\]
%4th Problem end

%5th Problem start
\item 
Find a particular solution of the differential equation:
\[ y'' - 3y' + 2y = e^{3x}(-1 + 2x + x^2) \]
and plot it.\\

\emph{\textbf{Solution:}}
To find the particular solution of the differential equation, since the RHS is in the form of quadratic equation multiplied with an exponential term, we assume,
\[
y_p = (Ax^2 + Bx + C)e^{3x}
\]
differentiating once,
\[
{y_p}' = e^{3x}(2Ax + B) + 3e^{3x}(Ax^2 + Bx + C)
\]
differentiating twice
\[
{y_p}'' = 9Ae^{3x} + (12A + 9B)xe^{3x} + (2A + 6B + 9C)e^{3x} 
\]
substituting the values of ${y_p}'',{y_p}'$ and ${y}$, we get
\[
e^{3x}[2Ax^2 + (6A + 2B)x + 2A + 3B + 2C] = (x^2 + 2x - 1)e^{3x}
\]
Comparing the coefficients of $x^2,x$ and constant terms,
\[
A = 1/2,\;B = -1/2,\;C = -1/4
\]
Substituting the values in the ${y_p}$, we get
\[
{y_p}(x) = e^{3x}(\frac{x^2}{2} -\frac{x}{2} -\frac{1}{4})
\]
\begin{figure}[htbp]
    \centering
    \includegraphics[width=0.8\textwidth]{Images/Problem 5.jpg}
    \caption{Plot of $y_p(x) = e^{3x}(x^2/2 - x/2 - 1/4)$}
\end{figure}
%5th Problem end

%6th Problem start
\item 
Solve the following differential equation by undetermined coefficients
\[ y'' - 2y' + 5y = e^x \cos(2x) \]\\
\emph{\textbf{Solution:}}
The characteristic equation is
\[\lambda^2 - 2\lambda + 5 = 0\]
Solving the quadratic equation, we get \(\lambda_1 = 1 + 2i\) and \(\lambda_2 = 1 - 2i\).
Since the roots are complex, the general solution of the homogeneous ODE is
\[
y_c = e^{x}(C_1\,\cos(2\,x) + C_2\,\sin(2\,x))
\]
Since the RHS is a trigonometric function multiplied with an exponential function, we choose the particular solution as
\[
y_p = e^x(A\,\cos(2\,x) + B\,\sin(2\,x))
\]
Since we have $e^x$ in our solution for the homogeneous equation, we multiply $y_p$ with an $x$
\[
y_p = xe^x(A\,\cos(2\,x) + B\,\sin(2\,x))
\]
Differentiating once,
\[
y_{p}'=\left(\left(2\,B+A\right)\,x+A\right)\,{e}^{x}\,\cos\left(2\,x\right)+\left(\left(B-2\,A\right)\,x+B\right)\,{e}^{x}\,\sin\left(2\,x\right)
\]
Differentiating twice,
\[
y_{p}''=\left(\left(4\,B-3\,A\right)\,x+4\,B+2\,A\right)\,{e}^{x}\,\cos\left(2\,x\right)+\left(\left(-3\,B-4\,A\right)\,x+2\,B-4\,A\right)\,{e}^{x}\,\sin\left(2\,x\right)
\]
Substituting ${y_p}'', {y_p}'$ and ${y}$ in the differential equation
\[
4\,B\,{e}^{x}\,\cos\left(2\,x\right)-4\,A\,{e}^{x}\,\sin\left(2\,x\right)={e}^{x}\,\cos\left(2\,x\right)
\]
Comparing the coefficients to find $A$ and $B$,
\[
A = 0,\,and\,B=\frac{1}{4}
\]
Substituting the values of $A$ and $B$ to the particular solution,
\[
y_{p}=\dfrac{x\,{e}^{x}\,\sin\left(2\,x\right)}{4}
\]
The general solution is $y = y_c + y_p$,
\[
{y}(x) = C_{1}\,{e}^{x}\,\cos\left(2\,x\right)+C_{2}\,{e}^{x}\,\sin\left(2\,x\right)+\dfrac{x\,{e}^{x}\,\sin\left(2\,x\right)}{4}
\]

%7th Problem starts
\item 
Find the frequency of oscillation of a pendulum of length $L$, neglecting air resistance and the weight of the rod, and assuming the amplitude of oscillation $\theta$ to be so small that $\sin \theta$ practically equals $\theta$.\\

\emph{\textbf{Solution:}}
The equation of motion for a simple pendulum can be given as:
\[ \frac{d^2\theta}{dt^2} + \frac{g}{L} \sin{\theta} = 0 \]
where $g$ is the acceleration due to gravity and $L$ is the length of the pendulum.

Assuming a small amplitude of oscillation, we can approximate $\sin \theta \approx \theta$. Therefore, the equation becomes:
\[ \frac{d^2\theta}{dt^2} + \frac{g}{L} \theta = 0 \]

This is a second-order linear homogeneous differential equation with constant coefficients. Its characteristic equation is:
\[ \lambda^2 + \frac{g}{L} = 0 \]
Solving for $\lambda$, we find two complex roots:
\[ \lambda = \pm i \sqrt{\frac{g}{L}} \]

The general solution for the differential equation:
\[ \theta(t) = A \cos \left( \sqrt{\frac{g}{L}} t \right) + B \sin \left( \sqrt{\frac{g}{L}} t \right) \]

The frequency of oscillation $f$, is related to the angular frequency $\omega$ by the formula:
\[ \omega = 2\pi f \]
For the pendulum, $\omega$ is:
\[ \omega = \sqrt{\frac{g}{L}} \]
The frequency of oscillation is:
\[f = \frac{1}{2\pi} \sqrt{\frac{g}{L}} \]
%7th Problem ends

%8th Problem starts
\item 
Show that the ratio of two consecutive maximum amplitudes of a damped free oscillation governed by the equation
\[ m y'' + c y' + k y = 0 \]
is constant. Show also that the natural logarithm of this ratio, called the logarithmic decrement, is given by $\Delta = \frac{2\pi \alpha}{\omega^*}$
where
\[ \alpha = \frac{c}{2m}\; and\; \omega^* = \frac{\sqrt{4mk - c^2}}{2m} \]

Find $\Delta$ for the solutions of
\[ {y}'' + 2{y}' + 5y = 0 \]\\
\emph{\textbf{Solution:}}
If the system is under-damped it will swing back and forth with decreasing size of the swing until it comes to a stop. Its amplitude will decrease exponentially.
\\
The phase angle form of the amplitude is given by
\[
x(t) = C\,e^{-\alpha\,t}\cos(\omega^*\,t - \phi)
\]
let $t_0$ and $t_1$ be the time between two maximum amplitudes, 
\[
x(t_0) = C\,e^{-\alpha\,t_0}\cos(\omega^*\,t_0 - \phi)
\]
\[
x(t_1) = C\,e^{-\alpha\,t_1}\cos(\omega^*\,t_1 - \phi)
\]
we know that the time period(T) is,
\[
T = \frac{2\pi}{\omega^*}
\]
So,
\[
t_1 = t_0 +\frac{2\,\pi}{\omega^*} 
\]
substituting in $x({t_1})$,we get
\[
x(t_1) = C\,e^{-\alpha\,t_1}\cos(\omega^*\,t_0 - \phi + 2\pi)
\]
Dividing $x({t_0})$ by $x({t_1})$, we get
\[
\frac{x({t_0})}{x({t_1})} = e^{\frac{2 \pi \alpha}{\omega^*}}
\]
So, the ratio of two consecutive maximum amplitudes is constant.\\
Taking natural logarithm of this ratio,
\[
\Delta = \ln{\frac{x({t_0})}{x({t_1})}} = \frac{2 \pi \alpha}{\omega^*}
\]
To find the $\Delta$ of the given ODE ${y}'' + 2{y}' + 5y = 0$,
\[
m = 1,\,c = 2,\,k = 5
\]
Substituting and solving for $\omega^*$, we get,
\[
\omega^* = 2
\]
Substituting the values in $\Delta$, we get,
\[
\Delta = \pi
\]
%8th problem ends
%9th Problem starts
\item 
How does the frequency of the harmonic oscillation change if we:
(i) Double the mass?
(ii) Take a spring of twice the modulus? First find qualitative answers by physics, then look at formulas.\\

\emph{\textbf{Solution:}}

Qualitative answer by physics:

1. When the mass is doubled, the inertia of the system increases. This means that it will be more resistant to acceleration and deceleration. As a result, the system's natural frequency will decrease. In other words, it will oscillate more slowly.

2. When the spring modulus is doubled, it indicates that the spring becomes stiffer. A stiffer spring will exert a stronger force for a given displacement from the equilibrium position. This increased force will lead to a higher acceleration of the mass, causing it to oscillate more quickly. Therefore, doubling the spring constant will result in an increase in the natural frequency of the system.

From formulas,When\\
(i) Doubling the mass?\\
Let $F_1$ be the original frequency and $F_2$ be the frequency after doubling the mass,
\[
F = \frac{\omega_0}{2 \pi}\\
\]
\[
F_1 = \frac{\sqrt{\frac{k}{m_1}}}{2 \pi}\\
\]
\[
m_2 = 2\,m_1
\]
\[
F_2 = \frac{\sqrt{\frac{k}{m_2}}}{2 \pi}\\
\]
Substituting $m_2$, we get,
\[
F_1 = \sqrt{2}\,F_2
\]
After doubling the mass, frequency is lowered by the factor of $\sqrt{2}$.\\

(ii) Take a spring of twice the modulus?
\[
k_2 = 2\,k_1
\]
\[
F_1 = \frac{F_2}{\sqrt{2}}
\]
After doubling the modulus, frequency is higher by the factor of $\sqrt{2}$.
%9th Problem ends
%10th Problem starts
\item 
The natural length of a spring is 1 m. An object is attached to it, and the length of the spring increases to 102 cm when the object is in equilibrium. Then the object is initially displaced downward 1 cm and given an upward velocity of 14 cm/s. Find the displacement for \( t > 0 \).
Also, find the natural frequency, period, amplitude, and phase angle of the resulting motion.\\

\emph{\textbf{Solution:}}\\
Given data:
$l_1 = 100\, cm$ and $l_2 = 102\,cm$, $\Delta l = 2\,cm$,\,${y}'(0) = 14,\,y(0) = -1$\\
The differential equation can be written as,
\[
{y}'' + 490y = 0
\]
The auxiliary equation is
\[
\lambda^2 + 490 = 0
\]
Solving for $\lambda$,
\[
\lambda_1 = 7\sqrt{10}i,\,\lambda_2 = -7\sqrt{10}i
\]
The general solution,
\[
{y}(t) = C_1\,\cos{7\sqrt{10}\,t} + C_2\,\sin{7\sqrt{10}\,t}
\]
Differentiating and applying the initial conditions,
\[
{y}'(t) = 7\,\sqrt{10}\,(-C_1\,\sin{7\sqrt{10}\,t} + C_2\,\cos{7\sqrt{10}\,t})
\]
\[
C_1 = -1\,\, and\,\, C_2 = \frac{2}{\sqrt{10}}
\]
The frequency is $7\sqrt{10}\, rad/s$, and the time period is $T = \frac{2\,\pi}{(7\,\sqrt{10})}\, s$.\\
The amplitude is\\
\[
R = \sqrt{{C_1}^2 + {C_2}^2}
\]
\[
R = \sqrt{{-1}^2 + {\frac{2}{\sqrt{10}}}^2}
\]
\[
R = \sqrt{\frac{7}{5}}\,\,cm.
\]
The phase angle,
\[
\cos{\phi} = - \sqrt{\frac{5}{7}}\,\, and\,\, \sin{\phi} = \sqrt{\frac{2}{7}}
\]
\[
\phi = \cos^{-1}{(-\sqrt{\frac{5}{7}})}
\]
\[
\phi \approx 2.58\,radians.
\]
%10th Problem ends
%11th Problem starts
\item 
The mass ($m = 0.1 \, \text{kg}$) attached to a spring ($k = 50 \, \text{N/m}$) is initially stretched to $1 \, \text{m}$ length from the equilibrium position and released without any initial velocity. The motion is damped ($c = 2 \, \text{kg/s}$) and is being driven by an external periodic ($T = \frac{\pi}{10} \, \text{s}$) force beginning at $t = 0$. Determine the displacement of mass as a function of time. Plot the displacement versus time graph.\\

\emph{\textbf{Solution:}}\\
Given data:
$m = 0.1 \text{kg}, \, c = 2 \, \text{kg/s}, \, k = 50 \, \text{N/m}, \, T = \frac{\pi}{10} \, \text{s}, \, {x}(0) = 1 \, {x}'(0) = 0,$
$F_0 = 50 \, \text{N}$.\\
From T, we can get $\omega = 20$
The differential equation can be written as,
\[
0.1{x}'' + 2{x}' + 50{x} = 50 \, \cos{20\,t}
\]
Dividing the equation by 0.1,
\[
{x}'' + 20{x}' + 500{x} = 500 \, \cos{20\,t}
\]
The characteristic equation is
\[\lambda^2 + 20\lambda + 500 = 0\]
Solving the quadratic equation, we get \(\lambda_1 = -10 + 20i\) and \(\lambda_2 = -10 - 20i\).
Since the roots are complex, the general solution of the homogeneous ODE is
\[
x_c = e^{-10\,t}(C_1\,\cos(20\,t) + C_2\,\sin(20\,t))
\]
Since the RHS is a trigonometric function, we choose the particular solution as
\[
x_p = A\,\cos(20\,t) + B\,\sin(20\,t)
\]
differentiating once,
\[
x_{p}'= -20\,A\,\sin\left(20\,t\right) + 20\,B\,\cos\left(20\,t\right)
\]
differentiating twice
\[
x_{p}''= -400\,A\,\cos\left(20\,t\right)-400\,B\,\sin\left(20\,t\right)
\]
substituting the values of ${x_p}'',{x_p}'$ and ${x}$, we get
\[
(100\,A + 400\,B)\,\cos{20\,t} + (-400\,A + 100\,B) = 20\,\cos{20\,t}
\]
Comparing the coefficients of $\cos{20\,t}$ and $\sin{20\,t}$,
\[
A = \frac{5}{17}\,\,\,and\,\,\, B = \frac{20}{17} 
\]
Substituting the values in the ${x_p}$, we get
\[
x_p = (\frac{5}{17})\,\cos{20\,t} + (\frac{20}{17})\,\sin{20\,t}
\]
The general equation is,
\[
x(t) = e^{-10\,t}\,(C_1\,\cos{20\,t} + C_2\,\sin{20\,t}) + (\frac{5}{17})\,\cos{20\,t} + (\frac{20}{17})\,\sin{20\,t}
\]
Differentiating the general solution,
\[
{x}'(t) = e^{-10\,t}\,(-20\,C_1\,\sin{20\,t} + 20\,C_2\,\cos{20\,t}) -\,10\,e^{-10\,t}\,(C_1\,\cos{20\,t} + C_2\,\sin{20\,t}) - \frac{100}{17}\,\sin{20\,t} + \frac{400}{17}\,\cos{20\,t}
\]
Substituting the initial conditions and solving for $C_1$\, and\, $C_2$,
\[
C_1 = \frac{12}{17}\,\,\,and\,\,\, C_2 = \frac{-14}{17}
\]
Substituting the values in the general solution,
\[
x(t) = e^{-10\,t}\,((\frac{12}{17})\,\cos{20\,t} - (\frac{14}{17})\,\sin{20\,t}) + (\frac{5}{17})\,\cos{20\,t} + (\frac{20}{17})\,\sin{20\,t}
\]
\begin{figure}[htbp]
    \centering
    \includegraphics[width=0.6\textwidth]{Images/Problem 11.jpg}
    \caption{Plot of Displacement vs Time}
\end{figure}
\pagebreak
%11th Problem ends
%12th Problem starts
\item 
Solve the initial value problem
\[
\frac{d^2x}{dt^2} + \omega_0^2x = \cos(\omega t),\,\,\,\, {x}(0) = 0,\,\,\, {x}'(0) = 0
\]
for $\omega_0 = 50$, $\omega = 20$, $40$, $48$, $52$, $50$, where $\omega_0$ and $\omega$ are in rad/s. Plot the graph $x$ vs $t$ for each of these cases.\\

\emph{\textbf{Solution:}}\\
From the differential equation,
$m = 1$\,\, $k = {\omega_0}^2$,\,\, ${x}(0) = 0$,\,\, ${x}'(0) = 0$,\,\, $\omega_0 = 50$,\,\, $F_0 = 1$

The characteristic equation is
\[\lambda^2 + {\omega_0}^2 = 0\]
Solving the quadratic equation, we get \(\lambda_1 = {\omega_0}\,i\) and \(\lambda_2 = -{\omega_0}\,i\).
Since the roots are complex, the general solution of the homogeneous ODE is
\[
x_c = C_1\,\cos({\omega_0}\,t) + C_2\,\sin({\omega_0},t)
\]
Since the RHS is a trigonometric function, we choose the particular solution as
\[
x_p = (A\,\cos(\omega\,t) + B\,\sin(\omega\,t))
\]
differentiating once,
\[
x_{p}'= -\omega\,A\,\sin(\omega\,t) + \omega\,B\,\cos(\omega\,t)
\]
differentiating twice,
\[
x_{p}''= -\omega^2\,A\,\cos{(\omega\,t)} - \omega\,B\,\sin{(\omega\,t)}
\]
Substituting and comparing coefficients,
\[
(-\omega^2\,A\, + \omega^2\,AB)\,\cos{(\omega\,t)} + (-\omega^2\,B\, + \omega^2\,B)\,\sin{(\omega\,t)} = \cos{(\omega\,t)}
\]
\[
A = \frac{1}{{\omega_0}^2 - {\omega^2}}
\]
Substituting in $x_p$,
\[
x_p = (\frac{\cos{\omega\,t}}{{\omega_0}^2 - {\omega^2}})
\]
General solution ${x}(t) = {x_c} + {x_p}$,
\[
{x}(t) = C_1\,\cos({\omega_0}\,t) + C_2\,\sin({\omega_0},t) +  (\frac{\cos({\omega\,t})}{{\omega_0}^2 - {\omega^2}})
\]
Differentiating the general solution,
\[
{x}'(t) = -C_1\,\omega_0\,\sin({\omega_0}\,t) + C_2\,\omega_0\,\cos({\omega_0},t) -  (\frac{\omega\,\sin({\omega\,t})}{{\omega_0}^2 - {\omega^2}})
\]
Applying the initial conditions ${x}(0) = 0$,\,\, ${x}'(0) = 0$\, and solving for $C_1$\, and\, $C_2$
\[
C_1 = -\,(\frac{1}{({\omega_0}^2 - {\omega^2}})
\]
General solution,
\[
{x}(t) = \frac{\cos{(\omega\,t)} - \cos{(\omega_0\,t)} }{{\omega_0}^2 - {\omega^2}}
\]
Substituting $\omega_0 = 50$,
\[
{x}(t) = \frac{\cos{(\omega\,t)} - \cos{(50\,t)} }{2500 - {\omega^2}}
\]
When $\omega = 20$,
\[
{x}(t) = \frac{1}{2100}\,(\cos{20\,t} - \cos{50\,t})
\]
When $\omega = 40$,
\[
{x}(t) = \frac{1}{900}\,(\cos{40\,t} - \cos{50\,t})
\]
When $\omega = 48$,
\[
{x}(t) = \frac{1}{196}\,(\cos{48\,t} - \cos{50\,t})
\]
When $\omega = 52$,
\[
{x}(t) = \frac{-1}{204}\,(\cos{52\,t} - \cos{50\,t})
\]
When $\omega = 50$,\, i.e\,$\omega = \omega_0$ The solution becomes, 
\[
{x}(t) = (\frac{1}{2\,\omega_0})\,t\,\sin{(\omega_0\,t)}
\]
Because of the factor t, the amplitude of the vibration becomes larger and larger. Practically the system would fail.

\begin{figure}[htbp]
    \centering
    \includegraphics[width=0.8\textwidth]{Images/Problem 12 - 20.jpg}
    \caption{Displacement of x when $\omega = 20$}
\end{figure}
\begin{figure}[htbp]
    \centering
    \includegraphics[width=0.8\textwidth]{Images/Problem 12 - 40.jpg}
    \caption{Displacement of x when $\omega = 40$}
\end{figure}
\begin{figure}[htbp]
    \centering
    \includegraphics[width=0.8\textwidth]{Images/Problem 12 - 48.jpg}
    \caption{Displacement of x when $\omega = 48$}
\end{figure}
\begin{figure}[htbp]
    \centering
    \includegraphics[width=0.8\textwidth]{Images/Problem 12 - 52.jpg}
    \caption{Displacement of x when $\omega = 52$}
\end{figure}
\begin{figure}[t]
    \centering
    \includegraphics[width=0.8\textwidth]{Images/Problem 12 - 50.jpg}
    \caption{Displacement of x when $\omega = 50$}
\end{figure}
\end{enumerate}
\end{document}